\documentclass[10pt,a4paper]{article}
\usepackage[utf8]{inputenc}
\usepackage[russian]{babel}
\usepackage[OT1]{fontenc}
\usepackage{amsmath}
\usepackage{amsfonts}
\usepackage{amssymb}
\usepackage{graphicx}
\author{GSench}
\title{Численное решение Уравнения Переноса в одномерном случае при переменном поле скоростей}
\begin{document}
\section{Уравнение Переноса}
Уравнение переноса в общем случае имеет следующий вид:
\begin{equation}
\frac{\partial f}{\partial t}+\nabla\cdot(f\overrightarrow{u})=0
\end{equation}
Где u – векторное поле скоростей, f - переносимая скалярная величина, $ \nabla $ – оператор дивергенции. Определим f как функцию Хевисайда, принимающую значения 0 и 1:
\begin{equation}
f(x,t)=\begin{cases}1 & {\bf x}\in liquid \\0 & {\bf x}\not\in liquid\end{cases}
\end{equation}
В одномерном случае уравнение сводится к виду:
\begin{equation}
\label{eqn:TE1D}
\frac{\text{d}f_{x}}{\text{d}t} + \frac{\text{d}(f_{x}u_{x})}{\text{d}x}=0
\end{equation}
В дальнейшем будем обозначать $f_x$ $u_x$ просто как $f$ и $u$, подразумевая значения, взятые вдоль направлений соответствующих осей.

\section{Численное решение}
Для численного решения проводится дискретизация:

Отрезок $[0;X]$, на котором рассматривается данное уравнение, разбивается на $cellCount$ последовательных подотрезков, длиной $\Delta x_i$ каждый – ячейки сетки. i=1..cellCount. Положения $x_{i-\frac{1}{2}}$, $x_{i+\frac{1}{2}}$ являются узлами данной сетки (ребрами ячеек). $\Delta x_i=x_{i+\frac{1}{2}}-x_{i-\frac{1}{2}}$. Для реализации программы была выбрана равномерная сетка с ячейками равной длины $\Delta x$. Зададим длину временного шага $\Delta t$ и построим схему для вычисления средних значений функции $f(x,t)$ в каждой ячейке.
\begin{equation}
\overline{f}_i^n=\frac{1}{\Delta x_i}\int_{x_{i-\frac{1}{2}}}^{x_{i+\frac{1}{2}}}f(x,t_n)dx 
\end{equation}
- среднее по ячейке значение функции $f(x,t)$ на i-ом отрезке $\Delta x_i$ на n-ом временном шаге. В дальнейшем будем обозначать его как просто $f_i^n$.

Проинтегрируем уравнение переноса в одномерном случае (\ref{eqn:TE1D}) по времени на шаге $[t_n; t_{n+1}]$:
\[
(f^{n+1}-f^n)+\int_{t_n}^{t_{n+1}}\frac{\text{d}(f u)}{\text{d}x}d\tau=0
\]
Для численного дифференцирования используем явную разностную схему 2 порядка:
\[
\frac{\text{d}(f u)}{\text{d}x}=\frac{f_{i+\frac{1}{2}}u_{i+\frac{1}{2}}-f_{i-\frac{1}{2}}u_{i-\frac{1}{2}}}{\Delta x_i}
\]
В результате уравнение переноса преобразуется к виду:
\begin{equation}
(f_i^{n+1}-f_i^n)
+
\frac{1}{\Delta x_i}
\int_{t_n}^{t_{n+1}}
f_{i+\frac{1}{2}}u_{i+\frac{1}{2}}d\tau
-
\frac{1}{\Delta x_i}
\int_{t_n}^{t_{n+1}}
f_{i-\frac{1}{2}}u_{i-\frac{1}{2}} d\tau
=0
\end{equation}

\section{Поле скоростей твердого тела}
Поле скоростей твердого тела связано по формуле Эйлера:
\[
\overrightarrow{u}(\overrightarrow{r},t)=\overrightarrow{v_c}(t)+[\overrightarrow{\omega}(t)\times \overrightarrow{r}(t)]
\]
А значит, для любой его компоненты (по направлению соответствующей оси) $u(x,t)$ - линейная функция по координате x.Зафиксируем шаг по времени $t_n$ и найдем функцию $u(x)$:\\
Исходя из численного решения уравнения Эйлера интегрированием методом квадратур, нам известны значения $u_{i\pm\frac{1}{2}}$ на границах ячеек. Обозначим значения скорости и координаты на границах ячейки $\Delta x_i$:
\[ u_{i-\frac{1}{2}}=u_L \]
\[ u_{i+\frac{1}{2}}=u_R \]
А сами границы ячейки:
\[ x_{i-\frac{1}{2}}=x_L \]
\[ x_{i+\frac{1}{2}}=x_R \]
Тогда согласно тому, что u(x) - линейная функция:
\[
\frac{u-u_L}{u_R-u_L}=\frac{x-x_L}{x_R-x_L}
\]
Функция скорости на ячейке $\Delta x_i$ задается как:
\begin{equation}
u(x)=kx+b
\end{equation}
где
\[ k=\frac{u_R-u_L}{\Delta x} \]
\[ b=u_L-\frac{u_R-u_L}{\Delta x}x_L \]

\section{Метод характеристик}
Рассмотрим решение уравнения переноса с помощью метода характеристик.
\begin{equation}
\frac{\text{d}f}{\text{d}t} + \frac{\text{d}(f u)}{\text{d}x}=0
\end{equation}
Нам бы хотелось свести это дифференциальное уравнение в частных производных первого порядка к обыкновенному дифференциальному уравнению вдоль соответствующей кривой, то есть получить уравнение вида:
\[
\frac{\text{d}}{\text{d}s}f(x(s),t(s))=F(f,x(s),t(s))
\]
где кривая $(x(s),t(s))$ — характеристика.\\
Установим, что
\[
\frac{\text{d}}{\text{d}s}u(x(s),t(s))=\frac{\partial u}{\partial x}\frac{\text{d}x}{\text{d}s}+\frac{\partial u}{\partial t}\frac{\text{d}t}{\text{d}s}
\]



\end{document}