\documentclass[10pt,a4paper]{article}
\usepackage[utf8]{inputenc}
\usepackage{comment}
\usepackage[russian]{babel}
\usepackage[OT1]{fontenc}
\usepackage{amsmath}
\usepackage{amsfonts}
\usepackage{amssymb}
\usepackage{graphicx}
\author{GSench}
\title{Численное решение Уравнения Переноса в одномерном случае при переменном поле скоростей}
\begin{document}
\section{Уравнение Переноса}
Уравнение переноса в общем случае имеет следующий вид:
\begin{equation}
\frac{\partial f}{\partial t}+\nabla\cdot(f\overrightarrow{u})=0
\end{equation}
Где u – векторное поле скоростей, f - переносимая скалярная величина, $ \nabla $ – оператор дивергенции. Определим f как функцию Хевисайда, принимающую значения 0 и 1:
\begin{equation}
f(x,t)=\begin{cases}1 & {\bf x}\in liquid \\0 & {\bf x}\not\in liquid\end{cases}
\end{equation}
В одномерном случае уравнение сводится к виду:
\begin{equation}
\label{eqn:TE1D}
\frac{\text{d}f_{x}}{\text{d}t} + \frac{\text{d}(f_{x}u_{x})}{\text{d}x}=0
\end{equation}
В дальнейшем будем обозначать $f_x$ $u_x$ просто как $f$ и $u$, подразумевая значения, взятые вдоль направлений соответствующих осей.

\section{Численное решение}
Для численного решения проводится дискретизация:

Отрезок $[0;X]$, на котором рассматривается данное уравнение, разбивается на $cellCount$ последовательных подотрезков, длиной $\Delta x_i$ каждый – ячейки сетки. i=1..cellCount. Положения $x_{i-\frac{1}{2}}$, $x_{i+\frac{1}{2}}$ являются узлами данной сетки (ребрами ячеек). $\Delta x_i=x_{i+\frac{1}{2}}-x_{i-\frac{1}{2}}$. Для реализации программы была выбрана равномерная сетка с ячейками равной длины $\Delta x$. Зададим длину временного шага $\Delta t$ и построим схему для вычисления средних значений функции $f(x,t)$ в каждой ячейке.
\begin{equation}
\overline{f}_i^n=\frac{1}{\Delta x_i}\int_{x_{i-\frac{1}{2}}}^{x_{i+\frac{1}{2}}}f(x,t_n)dx 
\end{equation}
- среднее по ячейке значение функции $f(x,t)$ на i-ом отрезке $\Delta x_i$ на n-ом временном шаге. В дальнейшем будем обозначать его как просто $f_i^n$.

Проинтегрируем уравнение переноса в одномерном случае (\ref{eqn:TE1D}) по времени на шаге $[t_n; t_{n+1}]$:
\[
(f^{n+1}-f^n)+\int_{t_n}^{t_{n+1}}\frac{\text{d}(f u)}{\text{d}x}d\tau=0
\]
Для численного дифференцирования используем явную разностную схему 2 порядка:
\[
\frac{\text{d}(f u)}{\text{d}x}=\frac{f_{i+\frac{1}{2}}u_{i+\frac{1}{2}}-f_{i-\frac{1}{2}}u_{i-\frac{1}{2}}}{\Delta x_i}
\]
В результате уравнение переноса преобразуется к виду:
\begin{equation}
(f_i^{n+1}-f_i^n)
+
\frac{1}{\Delta x_i}
\int_{t_n}^{t_{n+1}}
f_{i+\frac{1}{2}}u_{i+\frac{1}{2}}d\tau
-
\frac{1}{\Delta x_i}
\int_{t_n}^{t_{n+1}}
f_{i-\frac{1}{2}}u_{i-\frac{1}{2}} d\tau
=0
\end{equation}

\section{Поле скоростей твердого тела}
Поле скоростей твердого тела связано по формуле Эйлера:
\[
\overrightarrow{u}(\overrightarrow{r},t)=\overrightarrow{v_c}(t)+[\overrightarrow{\omega}(t)\times \overrightarrow{r}(t)]
\]
Рассмотрим составляющие поля скоростей вдоль направлений X,Y,Z:
\[ \overrightarrow{u}(\overrightarrow{r},t)=(u_x, u_y, u_z)(\overrightarrow{r},t) \]
\[ \overrightarrow{r}(t)=(x-x_c(t), y-y_c(t), z-z_c(t)) \]
Векторы $\overrightarrow{v_c}(t)$ и $\overrightarrow{\omega}(t)$ считаются заданными.
\[ \overrightarrow{v_c}(t)=(v_{cx}(t), v_{cy}(t), v_{cz}(t)) \]
\[ \overrightarrow{\omega}(t)=(\omega_x(t), \omega_y(t), \omega_z(t)) \]

\[ u_x(t)=v_{cx}(t)+\omega_y(t)z-\omega_y(t)z_c(t)-\omega_z(t)y+\omega_z(t)y_c(t) \]
\[ u_y(t)=v_{cy}(t)-\omega_x(t)z+\omega_x(t)z_c(t)+\omega_z(t)x-\omega_z(t)x_c(t) \]
\[ u_z(t)=v_{cz}(t)+\omega_x(t)y-\omega_x(t)y_c(t)-\omega_y(t)x+\omega_y(t)x_c(t) \]
Таким образом, можно видеть, что скорости твердого тела вдоль каждого из направлений не зависят от координаты рассматриваемой точки на этом направлении. То есть, например, скорости $u_x$ твердого тела вдоль направления $OX$ не зависят от координаты $x$, а лишь от положения $y$ и $z$ и от времени $t$.

В таком случае, при решении уравнения переноса мы можем рассматривать скорость постоянной вдоль каждого из направлений, но зависящей от времени. Поэтому вместо $u_{i-\frac{1}{2}}$, $u_{i+\frac{1}{2}}$ будем рассматривать $u$, зависящую от времени.

\section{Метод характеристик}
\begin{comment}
https://ru.wikipedia.org/wiki/%D0%9C%D0%B5%D1%82%D0%BE%D0%B4_%D1%85%D0%B0%D1%80%D0%B0%D0%BA%D1%82%D0%B5%D1%80%D0%B8%D1%81%D1%82%D0%B8%D0%BA
\end{comment}
Рассмотрим решение уравнения переноса с помощью метода характеристик.
\begin{equation}
\frac{\text{d}f}{\text{d}t} + \frac{\text{d}(f u)}{\text{d}x}=0
\end{equation}
Нам бы хотелось свести это дифференциальное уравнение в частных производных первого порядка к обыкновенному дифференциальному уравнению вдоль соответствующей кривой, то есть получить уравнение вида:
\[
\frac{\text{d}}{\text{d}s}f(x(s),t(s))=F(f,x(s),t(s))
\]
где кривая $(x(s),t(s))$ — характеристика.\\
Установим, что
\begin{equation}
\label{eqn:CharacteristicsMethodEq}
\frac{\text{d}}{\text{d}s}f(x(s),t(s))=\frac{\partial f}{\partial x}\frac{\text{d}x}{\text{d}s}+\frac{\partial f}{\partial t}\frac{\text{d}t}{\text{d}s}
\end{equation}
Положим, что
\[
\frac{\text{d}t}{\text{d}s}=1
\]
Следовательно, при $t(0)=0$, $s=t$. И теперь будем составлять ОДУ, используя метод характеристик в виде:
\[
\frac{\text{d}}{\text{d}t}f(x(t),t)=F(f,x(t),t)
\]
Будем искать решение вдоль характеристик, уравнение которых имеет вид:
\[
\frac{\text{d}x}{\text{d}t}=u(t)
\]
В таком случае уравнение (\ref{eqn:CharacteristicsMethodEq}) можно переписать в виде:
\[
\frac{\text{d}}{\text{d}t}f(x(t),t)=u(t)\frac{\partial f}{\partial x}+\frac{\partial f}{\partial t}
\]
таким образом, вдоль характеристики $(x(t),t)$ исходное уравнение в частных производных превращается в ОДУ:
\[
u_t=F(f,x(t),t)=0
\]
Данное уравнение говорит о том, что вдоль характеристик решение постоянное. Таким образом, $u(x,t)=u(x_0,0)$, где точки $(x,t)$ и $(x_0,0)$ лежат на одной характеристике. Видно, что для нахождения общего решения достаточно найти характеристики уравнения в виде:
\begin{equation}
\label{eqn:CharacteristicsEq}
\frac{\text{d}x}{\text{d}t}=u(t)
\end{equation}
Будем искать решение на слое $t_n$ по времени. $\tau$ - время на слое $[t_n; t_{n+1}]$. То есть $\tau=0 \Leftrightarrow t=t_n$, $\tau=\Delta t \Leftrightarrow t=t_{n+1}$.
\\Проинтегрируем уравнение (\ref{eqn:CharacteristicsEq}) по $t$ от $t_n$ до $t_n+\tau$:


\end{document}