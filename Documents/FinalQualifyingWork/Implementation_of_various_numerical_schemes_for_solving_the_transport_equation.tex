\documentclass[10pt,a4paper]{article}
\usepackage[utf8]{inputenc}
\usepackage[russian]{babel}
\usepackage[OT1]{fontenc}
\usepackage{amsmath}
\usepackage{amsfonts}
\usepackage{amssymb}
\usepackage{graphicx}
\author{GSench}
\title{Применение различных схем для численного решения уравнения переноса}
\begin{document}

\begin{titlepage}

\begin{center}
\textsc{Федеральное государственное бюджетное образовательное учреждение высшего образования\\
<<Московский государственный университет имени М.В. Ломоносова>>}\\
\vspace{12pt}
\textsc{Механико-математический факультет}\\
\textsc{Кафедра вычислительной механики}\\

\vspace*{\fill}
\textsc{Выпускная квалификационная работа\\
(Дипломная работа)\\
специалиста}\\
\vspace{12pt}
\textsc{\textbf{Применение различных схем для численного решения уравнения переноса}}\\
\vspace*{\fill}

\end{center}

\begin{flushright}
Студент 621 группы \\
Сенченок Григорий Антонович\\
\vspace{10pt}
Научный руководитель: \\
д.ф.-м.н., профессор Меньшов Игорь Станиславович
\end{flushright}

\mbox{}
\vfill
\begin{center}
Москва\\[5pt]
2022
\end{center}

\end{titlepage}

\tableofcontents

%\maketitle

\newpage

\section{Аннотация}

\section{Введение}

\section{Исследование литературы}

\section{Постановка задачи}

\subsection{Цифровая геометрия}
\subsection{Точное решение: Direct Motion}
\subsection{Уравнение переноса}
\subsection{Jump Reconstruction}

\section{Ограничения и допущения}

\section{Точное решение}
\subsection{Постановка задачи для точного решения}
\subsection{Теория}
\subsection{Численное решение}
\subsection{Исследование сходимости}

\section{Численное решение уравнения переноса}
\subsection{Уравнение переноса}
\subsection{Теория}
\subsection{Одномерный случай со статическим полем скоростей}
\subsection{Численное решение}
\subsection{Исследование сходимости}

\end{document}